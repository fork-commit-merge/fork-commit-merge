% LaTeX - Easy

\documentclass{article}
\usepackage{amsmath}

\title{Typesetting Mathematics in LaTeX}
\author{DEMAGA LAWRENCE}
\date{\today}

\begin{document}
\maketitle

% TODO: Add your content here
\section*{Fractions}
To typeset fractions in \LaTeX, we use the \verb|\frac{}{}| command.

For example, the equation:
\[
\frac{a}{b} + \frac{c}{d} = \frac{ad + bc}{bd}
\]
Here, \(a, b, c, d\) are variables, and the expression demonstrates the addition of two fractions with denominators \(b\) and \(d\).

\section*{Integrals}
Integrals are typeset using the \verb|\int| command. For example:
\[
\int_{a}^{b} f(x)\,dx
\]
In this equation:
\begin{itemize}
    \item \(a\) is the \textbf{lower limit of integration}.
    \item \(b\) is the \textbf{upper limit of integration}.
    \item \(f(x)\) is the \textbf{integrand}, and \(dx\) represents the variable of integration.
\end{itemize}

\section*{Matrices}
Matrices can be written using the \verb|bmatrix| environment:
\[
\begin{bmatrix}
a & b \\
c & d
\end{bmatrix}
\]
In this \(2 \times 2\) matrix:
\begin{itemize}
    \item \(a\) is the element in the first row, first column.
    \item \(b\) is the element in the first row, second column.
    \item \(c\) is the element in the second row, first column.
    \item \(d\) is the element in the second row, second column.
\end{itemize}

\end{document}
